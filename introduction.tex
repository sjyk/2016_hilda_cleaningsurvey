\section{Introduction}
Emphasize qualitative study

If we build tools that are super easy to use and super easy to iterate with, 

we’ll be helping the common man.